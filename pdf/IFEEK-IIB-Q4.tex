% pdfLaTeX + CJKutf8
% Preamble {{{
\documentclass[12pt,a4paper]{exam}
%\usepackage{luatexja}
% pdfLaTeX
\usepackage{CJKutf8}
\AtBeginDocument{\begin{CJK*}{UTF8}{ipxg}}
\AtEndDocument{\end{CJK*}}
%
\usepackage[T1]{fontenc}
\usepackage{lmodern}
\usepackage{graphicx} % for \begin{figure} environment
%%%%% Exam Package %%%%%%%%%%%%%%%%%%%%%%%%%%%%%%%%%%%%%%%%
\renewcommand{\subpartlabel}{[\thesubpart]} % e.g. [i]
\renewcommand{\thequestion}{Q\arabic{question}} % e.g. Q1
%%%%% Maths %%%%%%%%%%%%%%%%%%%%%%%%%%%%%%%%%%%%%%%%%%%%%%%
\usepackage{amssymb,amsmath,amsthm,enumerate,cases}
%\usepackage{mathtools}
\allowdisplaybreaks[4] %縦揃えする数式を次ページに送ることを許可する
%%%%% Showlabels package %%%%%%%%%%%%%%%%%%%%%%%%%%%%%%%%%%
%\usepackage{showkeys} % turn off for a final version
%%%%% Line spacing package %%%%%%%%%%%%%%%%%%%%%%%%%%%%%%%%
% used with \doublespacing \singlespacing \onehalfspacing \setstretch{x} where x can be 1.5.
\usepackage{setspace}
\setstretch{1}
%%%%% Margins %%%%%%%%%%%%%%%%%%%%%%%%%%%%%%%%%%%%%%%%%%%%%
\usepackage{geometry}
\geometry{left=25mm,right=25mm,top=35mm,bottom=35mm}
%%%%% Page Number %%%%%%%%%%%%%%%%%%%%%%%%%%%%%%%%%%%%%%%%
\footer{}{Page \thepage\ of \numpages}{}
% Preamble }}}
% Tile {{{
\begin{document}
\begin{center}
	\large{\textbf{IFEEK特別演習IIB: 2018年度}}
\end{center}
\begin{center}
	\large{\textbf{
		Problem Set 4: Applications of the $\boldsymbol{IS}$-$\boldsymbol{LM}$ Model
	}} 
\end{center}
\vspace{5mm}
% }}}
\begin{questions}
	\question Accoording to the $IS$-$LM$ model, what happens to
			the interest rate, income, consumption, and
			investment under the following circumstances?
			\begin{parts}
				\part The central bank increases the money supply.
				\part The government increases government purchases.
				\part The government increases taxes.
				\part The government increases government purchases
					and taxes by equal amounts.
			\end{parts}
	\question Consider the following economy.
			\begin{parts}
				\part The consumption function is given by
					\begin{equation*}
						C=200+0.75(Y-T),
						\qquad
						I=200-25r,
						\qquad
						G=T=100
					\end{equation*}
					For this economy, graph the $IS$ curve for the interest rate 
					$r$ ranging from 0 to 8.
				\part The money demand function in this economy is
					\[
						\left(\frac{M}{P}\right)^d = Y-100r.
					\]
					The money supply $M$ and the price level $P$ are given by
					\[
						M=1,000,\qquad
						P=2.
					\]
					For this economy, graph the $LM$ curve for $r$ ranging
					from 0 to 8.
				\part Find the equilibrium interest rate $r$ and the
					equilibrium level ofincome $Y$.
				\part Suppose that government purchases are raised
					from 100 to 150. How much does the $IS$ curve
					shift? What are the new equilibrium interest
					rate and level of income?
				\part Suppose instead that the money supply is
					raised from 1,000 to 1,200. How much does
					the $LM$ curve shift? What are the new equilibrium
					interest rate and level of income?
				\part With the initial values for monetary and fiscal
					policy, suppose that the price level rises from 2
					to 4. What happens? What are the new equilibrium interest
					rate and level of income?
			\end{parts}
	\question Suppose that the government wants to raise investment
			but keep output constant. In the $IS$-$LM$ model, what mix of
			monetary and fiscal policy will achieve this goal?
			In the early 1980s, the US. government cut taxes and ran a
			budget deficit while the Fed pursued a tight monetary policy.
			What effect should this policy mix have?
	\question Suppose that the economy is in the long-run equilibrium with full
			employment. Use the $IS$-$LM$ diagram to describe the short-run and
			long-run effects of the following changes on national income, the interest
			rate, the price level, consumption, investment, and real money balances.
			In considering the long-run effects, assume that the general price
			level increases/decreases if output is above/below its long-run
			level.
			\begin{parts}
				\part An increase in the money supply.
				\part An increase in government purchases.
				\part An increase in taxes.
			\end{parts}
	\question Suppose that the Bank of Japan is considering two alternative monetary policies:
			\begin{itemize}
				\item holding the money supply constant and letting the interest
					rate adjust, or
				\item adjusting the money supply to hold the
					interest rate constant.
			\end{itemize}
			In the $IS$-$LM$ model, which policy will better stabilize output
			under the following conditions?
			\begin{parts}
				\part All shocks to the economy arise from exogenous
					changes in the demand for goods and services.  
				\part All shocks to the economy arise from exogenous
					changes in the demand for money.
			\end{parts}
	%\question Explain why each of the following statements is true.
			%Discuss the impact of monetary and fiscal policy in each of
			%these special cases.
			%\begin{parts}
				%\part If investment does not depend on the interest
					%rate, the $IS$ curve is vertical.
				%\part If money demand does not depend on the
					%interest rate, the $LM$ curve is vertical.
				%\part If money demand does not depend on income,
					%the $LM$ curve is horizontal.
				%\part If money demand is extremely sensitive to the
					%interest rate, the $LM$ curve is horizontal.
			%\end{parts}
	\question Answer the following.
			\begin{parts}
				\part Using the $IS$-$LM$ model, derive the
					aggregate demand ($AD$) curve using diagrams.
				\part What happens to the $AD$ curve if government
					purchases increase/decrease?
				\part What happens to the $AD$ curve if money
					supply increase/decrease?
			\end{parts}
	\question In the following equations, $a>0$, $1>b>0$, $c>0$, $d>0$
			$e>0$, $f>$, $\bar{G}>0$, $\bar{T}>0$, and $\bar{M}>0$ are all parameters.
			\begin{alignat*}{2}
				\text{The Goods Market}&:&\qquad C&=a+b(Y-T),\quad I=c-dr,\quad
				G=\bar{G},\quad T=\bar{T}\\
				\text{The Money Market}&:&\qquad \frac{M}{P}&=eY-fr,\quad
				M=\bar{M}.
			\end{alignat*}
			Derive an equation for the aggregate demand.
\end{questions}
\end{document}
